% Options for packages loaded elsewhere
\PassOptionsToPackage{unicode}{hyperref}
\PassOptionsToPackage{hyphens}{url}
\PassOptionsToPackage{dvipsnames,svgnames,x11names}{xcolor}
%
\documentclass[
  a4paper,
]{article}

\usepackage{amsmath,amssymb}
\usepackage{iftex}
\ifPDFTeX
  \usepackage[T1]{fontenc}
  \usepackage[utf8]{inputenc}
  \usepackage{textcomp} % provide euro and other symbols
\else % if luatex or xetex
  \usepackage{unicode-math}
  \defaultfontfeatures{Scale=MatchLowercase}
  \defaultfontfeatures[\rmfamily]{Ligatures=TeX,Scale=1}
\fi
\usepackage{lmodern}
\ifPDFTeX\else  
    % xetex/luatex font selection
\fi
% Use upquote if available, for straight quotes in verbatim environments
\IfFileExists{upquote.sty}{\usepackage{upquote}}{}
\IfFileExists{microtype.sty}{% use microtype if available
  \usepackage[]{microtype}
  \UseMicrotypeSet[protrusion]{basicmath} % disable protrusion for tt fonts
}{}
\makeatletter
\@ifundefined{KOMAClassName}{% if non-KOMA class
  \IfFileExists{parskip.sty}{%
    \usepackage{parskip}
  }{% else
    \setlength{\parindent}{0pt}
    \setlength{\parskip}{6pt plus 2pt minus 1pt}}
}{% if KOMA class
  \KOMAoptions{parskip=half}}
\makeatother
\usepackage{xcolor}
\usepackage[paperwidth=8.00in,paperheight=10.00in,left=1.25in,textwidth=
5.25in,top=1.00in,textheight=8.25in]{geometry}
\setlength{\emergencystretch}{3em} % prevent overfull lines
\setcounter{secnumdepth}{5}
% Make \paragraph and \subparagraph free-standing
\ifx\paragraph\undefined\else
  \let\oldparagraph\paragraph
  \renewcommand{\paragraph}[1]{\oldparagraph{#1}\mbox{}}
\fi
\ifx\subparagraph\undefined\else
  \let\oldsubparagraph\subparagraph
  \renewcommand{\subparagraph}[1]{\oldsubparagraph{#1}\mbox{}}
\fi


\providecommand{\tightlist}{%
  \setlength{\itemsep}{0pt}\setlength{\parskip}{0pt}}\usepackage{longtable,booktabs,array}
\usepackage{calc} % for calculating minipage widths
% Correct order of tables after \paragraph or \subparagraph
\usepackage{etoolbox}
\makeatletter
\patchcmd\longtable{\par}{\if@noskipsec\mbox{}\fi\par}{}{}
\makeatother
% Allow footnotes in longtable head/foot
\IfFileExists{footnotehyper.sty}{\usepackage{footnotehyper}}{\usepackage{footnote}}
\makesavenoteenv{longtable}
\usepackage{graphicx}
\makeatletter
\def\maxwidth{\ifdim\Gin@nat@width>\linewidth\linewidth\else\Gin@nat@width\fi}
\def\maxheight{\ifdim\Gin@nat@height>\textheight\textheight\else\Gin@nat@height\fi}
\makeatother
% Scale images if necessary, so that they will not overflow the page
% margins by default, and it is still possible to overwrite the defaults
% using explicit options in \includegraphics[width, height, ...]{}
\setkeys{Gin}{width=\maxwidth,height=\maxheight,keepaspectratio}
% Set default figure placement to htbp
\makeatletter
\def\fps@figure{htbp}
\makeatother
% definitions for citeproc citations
\NewDocumentCommand\citeproctext{}{}
\NewDocumentCommand\citeproc{mm}{%
  \begingroup\def\citeproctext{#2}\cite{#1}\endgroup}
\makeatletter
 % allow citations to break across lines
 \let\@cite@ofmt\@firstofone
 % avoid brackets around text for \cite:
 \def\@biblabel#1{}
 \def\@cite#1#2{{#1\if@tempswa , #2\fi}}
\makeatother
\newlength{\cslhangindent}
\setlength{\cslhangindent}{1.5em}
\newlength{\csllabelwidth}
\setlength{\csllabelwidth}{3em}
\newenvironment{CSLReferences}[2] % #1 hanging-indent, #2 entry-spacing
 {\begin{list}{}{%
  \setlength{\itemindent}{0pt}
  \setlength{\leftmargin}{0pt}
  \setlength{\parsep}{0pt}
  % turn on hanging indent if param 1 is 1
  \ifodd #1
   \setlength{\leftmargin}{\cslhangindent}
   \setlength{\itemindent}{-1\cslhangindent}
  \fi
  % set entry spacing
  \setlength{\itemsep}{#2\baselineskip}}}
 {\end{list}}
\usepackage{calc}
\newcommand{\CSLBlock}[1]{\hfill\break\parbox[t]{\linewidth}{\strut\ignorespaces#1\strut}}
\newcommand{\CSLLeftMargin}[1]{\parbox[t]{\csllabelwidth}{\strut#1\strut}}
\newcommand{\CSLRightInline}[1]{\parbox[t]{\linewidth - \csllabelwidth}{\strut#1\strut}}
\newcommand{\CSLIndent}[1]{\hspace{\cslhangindent}#1}

\makeatletter
\@ifpackageloaded{bookmark}{}{\usepackage{bookmark}}
\makeatother
\makeatletter
\@ifpackageloaded{caption}{}{\usepackage{caption}}
\AtBeginDocument{%
\ifdefined\contentsname
  \renewcommand*\contentsname{Table of contents}
\else
  \newcommand\contentsname{Table of contents}
\fi
\ifdefined\listfigurename
  \renewcommand*\listfigurename{List of Figures}
\else
  \newcommand\listfigurename{List of Figures}
\fi
\ifdefined\listtablename
  \renewcommand*\listtablename{List of Tables}
\else
  \newcommand\listtablename{List of Tables}
\fi
\ifdefined\figurename
  \renewcommand*\figurename{Figure}
\else
  \newcommand\figurename{Figure}
\fi
\ifdefined\tablename
  \renewcommand*\tablename{Table}
\else
  \newcommand\tablename{Table}
\fi
}
\@ifpackageloaded{float}{}{\usepackage{float}}
\floatstyle{ruled}
\@ifundefined{c@chapter}{\newfloat{codelisting}{h}{lop}}{\newfloat{codelisting}{h}{lop}[chapter]}
\floatname{codelisting}{Listing}
\newcommand*\listoflistings{\listof{codelisting}{List of Listings}}
\makeatother
\makeatletter
\makeatother
\makeatletter
\@ifpackageloaded{caption}{}{\usepackage{caption}}
\@ifpackageloaded{subcaption}{}{\usepackage{subcaption}}
\makeatother
\ifLuaTeX
  \usepackage{selnolig}  % disable illegal ligatures
\fi
\usepackage{bookmark}

\IfFileExists{xurl.sty}{\usepackage{xurl}}{} % add URL line breaks if available
\urlstyle{same} % disable monospaced font for URLs
\hypersetup{
  pdftitle={Smart Choices for Measurement Models},
  pdfauthor={Pablo Rogers},
  colorlinks=true,
  linkcolor={blue},
  filecolor={Maroon},
  citecolor={Blue},
  urlcolor={Blue},
  pdfcreator={LaTeX via pandoc}}

\title{Smart Choices for Measurement Models}
\usepackage{etoolbox}
\makeatletter
\providecommand{\subtitle}[1]{% add subtitle to \maketitle
  \apptocmd{\@title}{\par {\large #1 \par}}{}{}
}
\makeatother
\subtitle{Executable Manuscript Tutorial for your Confirmatory Factor
Analysis in R Environment}
\author{Pablo Rogers}
\date{March 7, 2024}

\begin{document}
\maketitle

\bookmarksetup{startatroot}

\section*{Abstract}\label{abstract}

\markboth{Abstract}{Abstract}

This article aims to accomplish three objectives: first, to compile
guidelines for the application of Confirmatory Factor Analysis (CFA), a
widely utilized technique in applied social sciences; second, to
demonstrate how these guidelines can be practically implemented through
a real-world example; and third, to structure this narrative using tools
that promote reproducibility, replicability, and transparency of
results. To this end, we propose a solution in the form of a tutorial
article wherein the key decisions made in conducting a CFA are validated
through recent literature and presented within a dynamic document
framework. This framework enables readers to access the article's source
code, utilized data, analytical execution codes, and various reading
media. We anticipate that by employing this pedagogical approach,
developed entirely within an open environment (utilizing
Git/Github/RStudio/Quarto/R packages + lavaan/Docker), researchers
proficient in specific statistical techniques relevant to their domains
will adopt and disseminate this proposal, thereby benefiting their
colleagues.

\textbf{Keywords:} Confirmatory Factor Analysis, Structural Equation
Modeling, Internal Structure Validity, Open Science, \emph{lavaan}.

\bookmarksetup{startatroot}

\section{Introduction}\label{introduction}

Confirmatory Factor Analysis (CFA) is a key method for assessing the
validity of a measurement instrument through its internal structure
(Bandalos 2018; Hughes 2018; Sireci and Sukin 2013). Validity is
arguably the most crucial characteristic of a measurement model (Furr
2021), as it addresses the essential question of what measuring
instruments truly assess (Bandalos 2018). This concern is closely linked
with the classical definition of validity: the degree to which a test
measures what it claims to measure (Bandalos 2018; Furr 2021; Sireci and
Sukin 2013; Urbina 2014), aligning with the tripartite model still
embraced by numerous scholars (Widodo 2018).

The tripartite model of validity frames the concept using three
categories of evidence: content, criterion, and construct (Bandalos
2018). Content validity pertains to the adequacy and representativeness
of test items relative to the domain or objective under investigation
(Cohen, Schneider, and Tobin 2022). Criterion validity is the
correlation between test outcomes and a significant external criterion,
such as performance on another measure or future occurrences (Cohen,
Schneider, and Tobin 2022). Construct validity evaluates the test's
capacity to measure the theoretical construct it is intended to assess,
taking into account related hypotheses and empirical data (Cohen,
Schneider, and Tobin 2022).

Introduced in the American Psychological Association (APA) ``Standards
for Educational and Psychological Testing'' in 1966, the tripartite
concept of validity has been a cornerstone in the social sciences for
decades (Bandalos 2018). However, its fragmented and confusing nature
has led to widespread criticism, prompting a shift towards a more
holistic view of validity (Sireci and Sukin 2013). This evolution was
signified by the publication of the 1999 standards (AERA, APA, and NCME
1999), and further by the 2014 standards (AERA, APA, and NCME 2014),
which redefined test validity in terms of the interpretations and uses
of test scores (Furr 2021). Under this new paradigm, validation requires
diverse theoretical and empirical evidence, recognizing validity as a
unified concept -- construct validity -- encompassing various evidence
sources for evaluating potential interpretations of test scores for
specific purposes (Furr 2021; Urbina 2014).

Thus, key authorities in psychological assessment now define validity as
the degree to which evidence and theory support the interpretations of
test scores for their intended purposes (AERA, APA, and NCME 2014).
Validity involves a comprehensive evaluation of how well empirical
evidence and theoretical rationales uphold the conclusions and actions
derived from test scores or other assessment types (Bandalos 2018; Furr
2021; Urbina 2014).

According to APA guidelines (AERA, APA, and NCME 2014), five types of
validity evidence are critical: content, response process, association
with external variables, consequences of test use, and internal
structure. Content validity examines the extent to which test content
accurately represents the domain of interest exclusively (Furr 2021).
The response process refers to the link between the construct and the
specifics of the examinees' responses (Sireci and Sukin 2013). Validity
based on external variables concerns the test's correlation with other
measures or constructs expected to be related or unrelated to the
evaluated construct (Furr 2021). The implications of test use focus on
the positive or negative effects on the individuals or groups assessed
(Bandalos 2018).

Evidence based on internal structure assesses how well the interactions
among test items and their components align with the theoretical
framework used to explain the outcomes of the measurement instrument
(AERA, APA, and NCME 2014; Rios and Wells 2014). Sources of internal
structural validity evidence may include analyses of reliability,
dimensionality, and measurement invariance.

Reliability is gauged by internal consistency, reflecting i) the
reproducibility of test scores under consistent conditions and ii) the
ratio of true score variance to observed score variance (Rios and Wells
2014). Dimensionality analysis aims to verify if item interrelations
support the inferences made by the measurement model's scores, which are
assumed to be unidimensional (Rios and Wells 2014). Measurement
invariance confirms that item properties remain consistent across
specified groups, such as gender or ethnicity.

CFA facilitates the integration of these diverse sources to substantiate
the validity of the internal structure (Bandalos 2018; Flora and Flake
2017; Hughes 2018; Reeves and Marbach-Ad 2016; Rios and Wells 2014). In
the applied social sciences, researchers often have a theoretical
dimensional structure in mind (Sireci and Sukin 2013), and CFA is
employed to align the structure of the hypothesized measurement model
with the observed data (Rios and Wells 2014).

CFA constitutes a fundamental aspect of the covariance-based Structural
Equation Modeling (SEM) framework (CB-SEM) (Brown 2023; Harrington 2009;
Jackson, Gillaspy, and Purc-Stephenson 2009; Kline 2023; Nye 2022). SEM
is a prevalent statistical approach in the applied social sciences
(Hoyle 2023; Kline 2023), serving as a generalization of multiple
regression and factor analysis (Hoyle 2023). This methodology
facilitates the examination of complex relationships between variables
and the consideration of measurement error, aligning with the
requirements for measurement model validation (Hoyle 2023).

Applications of CFA present significant complexities (Crede and Harms
2019; Jessica K. Flake, Pek, and Hehman 2017; Jessica Kay Flake and
Fried 2020; Jackson, Gillaspy, and Purc-Stephenson 2009; Nye 2022;
Rogers 2024), influenced by data structure, measurement level of items,
research goals, and other factors. CFA can proceed smoothly in scenarios
involving unidimensional measurement models with continuous items and
large samples, but may encounter challenges, such as diminished SEM
flexibility, when dealing with multidimensional models with ordinal
items and small sample sizes (Rogers 2024).

This leads to an important question: Can certain strategies within CFA
applications simplify the process for social scientists seeking evidence
of validity in the internal structure of a measurement model? This
inquiry does not suggest that research objectives should conform to
quantitative methods. Rather, research aims guide scientific inquiry,
defining our learning targets and priorities. Quantitative methods serve
as tools towards these ends, not as objectives themselves. They
represent one among many tools available to researchers, with the
study's purpose dictating method selection (Pilcher and Cortazzi 2023).

However, as the scientific method is an ongoing journey of discovery,
many questions, especially in Psychometrics concerning measurement model
validation, remain open-ended. The lack of consensus on complex and
varied topics suggests researchers should opt for paths offering maximal
analytical flexibility, enabling exploration of diverse methodologies
and solutions while keeping research objectives forefront (Price 2017).

A recurrent topic in Factor Analysis (FA) is how to handle the
measurement level of scale items. Empirical studies (Rhemtulla,
Brosseau-Liard, and Savalei 2012; Robitzsch 2022, 2020) advocating for
the treatment of scales with five or more response options as continuous
variables have shown to enhance CFA flexibility and address validity
evidence for the internal structure. The FA literature acknowledges
methodological dilemmas faced when dealing with binary and/or ordinal
response items with fewer than five options (Rogers 2024, 2022).

For continuous scale items, the maximum likelihood (ML) estimator and
its robust variations are applicable. For non-continuous items,
estimators from the Least Squares (cat-LS) family are recommended (Nye
2022; Rogers 2024, 2022). Though cat-LS estimators impose fewer
assumptions on data, they require larger sample sizes, more
computational power, and greater researcher expertise (Robitzsch 2020).

Assessing model fit is more challenging with cat-LS estimated models
compared to those estimated by ML, which are better established and more
familiar to researchers (Rhemtulla, Brosseau-Liard, and Savalei 2012).
Despite their increasing popularity, cat-LS models are newer, less
recognized, and seldom available in software (Rhemtulla, Brosseau-Liard,
and Savalei 2012). Handling missing data remains straightforward with ML
models using the Full Information ML (FIML) method but is problematic
with ordinal data (Rogers 2024).

Thus, we can optimize the potential of some of the available software
(Arbuckle 2019; Bentler and Wu 2020; Fox 2022; JASP Team 2023; Jöreskog
and Sörbom 2022; Muthén and Muthén 2023; Neale et al. 2016; Ringle,
Wende, and Becker 2022; Rosseel 2012; The jamovi project 2023) and
overcome many of the limitations for ordinal and nominal data, which are
still present in some of them (Arbuckle 2019; Bentler and Wu 2020; Neale
et al. 2016; Ringle, Wende, and Becker 2022).

This discussion does not intend to oversimplify, digress, or claim
superiority of one software over another. Rather, it underscores a
fundamental statistical principle: transitioning from nominal to ordinal
and then to scalar measurement levels increases the flexibility of
statistical methods. Empirical studies in CFA support these
clarifications (Rhemtulla, Brosseau-Liard, and Savalei 2012; Robitzsch
2022, 2020).

This article assists applied social scientists in decision-making from
selecting a measurement model to comparing and updating models for
enhanced CFA flexibility. It addresses power analysis, data
preprocessing, estimation procedures, and model modification from three
angles: smart choices or recommended practices (Jessica K. Flake, Pek,
and Hehman 2017; Nye 2022; Rogers 2024), pitfalls to avoid (Crede and
Harms 2019; Rogers 2024), and essential reporting elements (Jessica Kay
Flake and Fried 2020; Jackson, Gillaspy, and Purc-Stephenson 2009;
Rogers 2024).

The aim is to guide researchers through CFA to access the underlying
structure of measurement models without falling into common traps at any
stage of the validation process. Early-stage decisions can preempt later
limitations, while missteps may necessitate exploratory research or
additional efforts in subsequent phases.

Practically, this includes an R tutorial utilizing the lavaan package
(Rosseel 2012), adhering to reproducibility, replicability, and
transparency standards of the Open Science movement (Gilroy and Kaplan
2019; Kathawalla, Silverstein, and Syed 2021; Klein et al. 2018).

Tutorial articles, following the FAIR principles (Findable, Accessible,
Interoperable, and Reusable) (Wilkinson et al. 2016), play a vital role
in promoting open science (Martins 2021; Mendes-Da-Silva 2023), by
detailing significant methods or application areas in an accessible yet
comprehensive manner. This encourages adherence to best practices among
researchers, minimizing the impact of positive publication bias.

This tutorial is structured into three sections, beyond the introductory
discussion. It includes a thorough review of CFA recommended practices,
an example of real-world research application in the R ecosystem, and
final considerations, following (\textbf{Martins2021?}) format for
tutorial articles. This approach, combined with workflow recommendations
for reproducibility, aims to support the applied social sciences
community in effectively utilizing CFA (Martins 2021; Mendes-Da-Silva
2023).

\bookmarksetup{startatroot}

\section{Smart Choices in CFA}\label{smart-choices-in-cfa}

This paper presents a comprehensive approach to conducting a standard
CFA within the applied social sciences, following the guidelines
outlined by Rogers (2024). According to Rogers (2024), a typical CFA
study seeks to fit a reflective common factor model with a predefined
multifactor structure, established psychometric properties, and a
maximum of five Likert-type response options. This scenario frequently
occurs in research endeavors where the measurement model facilitates the
examination of hypotheses derived from the structural model.

The initial phase in such research involves data preprocessing.
Specifically, for categorical data, Rogers (2024) advises employing
multiple imputation to handle missing data, taking into consideration
the limitations posed by available software and methodologies (Rogers
2024). When a measurement model allows for the treatment of items as
continuous variables, addressing this challenge can be deferred to the
estimation process stage through the selection of an appropriate
estimator (Robitzsch 2022).

This paper reinterprets the insights from Rogers (2024) for CFAs that
accommodate continuous item treatment. Thus, a strategic choice involves
opting for measurement models that permit this approach, thereby
circumventing methodological hurdles (Robitzsch 2022, 2020) associated
with binary and/or ordinal response items with up to four or five
gradations. Such a decision influences various aspects of the research
process, including the choice of software, power analysis, estimation
techniques, criteria for model adjustment, and model comparisons. These
choices, in turn, affect requirements concerning sample size,
computational resources, and the researcher's expertise (Robitzsch
2020).

Subsequent sections delve into themes previously summarized by Rogers
(2024), specifically concerning CFAs with ordinal items. These themes
are explored in terms of recommended practices (Jessica K. Flake, Pek,
and Hehman 2017; Nye 2022; Rogers 2024), pitfalls to avoid (Crede and
Harms 2019; Rogers 2024), and reporting guidelines (Jessica Kay Flake
and Fried 2020; Jackson, Gillaspy, and Purc-Stephenson 2009; Rogers
2024), all within the context of selecting measurement models that
accommodate continuous data interpretation.

Assuming that readers possess a foundational understanding of the topic,
this paper omits certain technical details, directing readers to
authoritative texts (Brown 2015; Kline 2023) and scholarly articles that
provide an introduction to Covariance-Based Structural Equation Modeling
(CB-SEM) (Davvetas et al. 2020; Shek and Yu 2014). The discussion is
framed within the CB-SEM paradigm (Brown 2015; Jackson, Gillaspy, and
Purc-Stephenson 2009; Kline 2023; Nye 2022), with a focus on CFA. The
paper explicitly excludes discussions on measurement model modifications
in Variance-Based SEM (VB-SEM), which are predominantly addressed in the
literature on Partial Least Squares SEM (PLS-SEM) (Hair et al. 2022,
2017; Henseler 2021).

\subsection{Measurement Model
Selection}\label{measurement-model-selection}

Selecting an appropriate measurement model is a critical initial step in
the research process. For robust analysis, it is advisable to prioritize
models that provide five or more ordinal response options. Research has
shown that a higher number of response gradations enhances the ability
to detect inaccurately defined models (Green et al. 1997;
Maydeu-Olivares, Fairchild, and Hall 2017), even when using estimators
designed for ordinal items (Xia and Yang 2018). This strategy also
mitigates some of the methodological challenges associated with the
analysis of ordinal data in CFA (Rhemtulla, Brosseau-Liard, and Savalei
2012; Robitzsch 2022, 2020).

When choosing a measurement scale, it is crucial to select ones that
have been validated in the language of application and with the study's
target audience (Jessica K. Flake, Pek, and Hehman 2017). Avoid scales
that are proprietary or specific to certain professions. An examination
of your country's Psychological Test Assessment System can be an
effective starting point. If the desired scale is not found within these
resources, consider looking into scales developed with the support of
public institutions, non-governmental organizations, research centers,
or universities, as these entities often invest significant resources in
validating measurement models for broader public policy purposes.

An extensive literature review is essential for selecting a suitable
measurement model. This should include consulting specialized journals,
books, technical reports, and academic dissertations or theses.
Schumacker, Wind, and Holmes (2021) provide a detailed guide for
initiating this search. Consideration should also be given to systematic
reviews or meta-analyses focusing on measurement models related to your
topic of interest. It is important to review both the original articles
on the scales and subsequent applications. Kline (2016) offers a useful
checklist for assessing various measurement methods.

Incorporate control questions, such as requiring respondents to select
``strongly agree'' on specific items, and monitor survey response times
to gauge participant engagement (Collier 2020).

Avoid adopting measurement models designed for narrow purposes or those
lacking rigorous psychometric validation (Jessica Kay Flake and Fried
2020; Kline 2016). The mere existence of a scale does not ensure its
validity (Jessica K. Flake, Pek, and Hehman 2017). Also, steer clear of
seldom-used or outdated scales, as they may have compromised
psychometric properties. Translating a scale from another language for
immediate use without thorough translation and retranslation processes
is inadvisable. Be cautious of overlooking alternative factorial
structures (e.g., higher-order or bifactor models) that could
potentially salvage the research if considered thoroughly (Crede and
Harms 2019).

When selecting a scale, justify its choice by highlighting its strong
psychometric properties, including previous empirical evidence of its
application within the target population and its reliability and
validity metrics (Jessica Kay Flake and Fried 2020; Jackson, Gillaspy,
and Purc-Stephenson 2009; Kline 2016). If the scale has multiple
potential factorial structures, provide a rationale for the chosen model
to prevent the misuse of CFA for exploratory purposes (Jackson,
Gillaspy, and Purc-Stephenson 2009).

Clearly specify the selected model and rationalize your choice by
detailing its advantages over other theoretical models. Illustrating the
models under consideration can further clarify your research approach
(Jackson, Gillaspy, and Purc-Stephenson 2009). Finally, identify and
explain any potential cross-loadings based on prior empirical evidence
(Brown 2023; Nye 2022), ensuring a comprehensive and well-justified
methodological foundation for your study.

\subsection{Power Analysis}\label{power-analysis}

When addressing Power Analysis (PA) in CFA and SEM, it's essential to
move beyond general rules of thumb for determining sample sizes.
Commonly cited guidelines suggesting minimum sizes or specific ratios of
observations to parameters (e.g., 50, 100, 200, 300, 400, 500, 1000 for
sample sizes or 20/1, 10/1, 5/1 for observation/parameter ratios) (Kline
2023; Kyriazos 2018) are based on controlled conditions that may not
directly transfer to your study's context.

Reliance on lower-bound sample sizes as a substitute for thorough PA
risks inadequate power for detecting meaningful effects in your model
(Westland 2010; Yilin Andre Wang 2023). Tools like Soper's calculator
(\url{https://www.danielsoper.com/statcalc/}), while popular and
frequently cited (as of 02/20/2024, with almost four years of existence,
it had collected more than 1,000 citations on Google Scholar), should
not replace a tailored PA approach. Such calculators, despite their
utility, may not fully accommodate the complexities and specific
requirements of your research design (Kyriazos 2018; Feng and Hancock
2023; Moshagen and Bader 2023).

A modern perspective on sample size determination emphasizes customizing
power calculations to fit the unique aspects of each study,
incorporating specific research settings and questions (Feng and Hancock
2023; Moshagen and Bader 2023). This approach underscores that there is
no universal sample size or minimum that applies across all research
scenarios (Kline 2023).

Planning for PA should ideally precede data collection, enhancing the
researcher's understanding of the study and facilitating informed
decisions regarding the measurement model based on existing literature
and known population characteristics (Feng and Hancock 2023; Leite,
Bandalos, and Shen 2023). A priori PA not only ensures adequate sample
size for detecting the intended effects, minimizing Type II errors, but
also aids in budgeting for data collection and enhancing overall
research design (Feng and Hancock 2023).

PA in SEM can be approached analytically, using asymptotic theory, or
through simulation methods. Analytical methods require specifying the
effect size in relation to the non-centrality parameter, while simulated
PA leverages a population model to empirically estimate power (Moshagen
and Bader 2023; Feng and Hancock 2023). These approaches are applicable
to assessing both global model fit and specific model parameters.

For CFA, evaluating the power related to the global fit of the
measurement model is recommended (Nye 2022). Although analytical
solutions have their limitations, they can serve as preliminary steps,
complemented by simulation techniques for a more comprehensive PA (Feng
and Hancock 2023; Moshagen and Bader 2023).

Several resources offer analytical solutions for global fit PA,
including ShinyApps by Jak et al. (2021), Moshagen and Bader (2023), Y.
Andre Wang and Rhemtulla (2021), and Zhang and Yuan (2018), with the
last application providing a comprehensive suite for Monte Carlo
Simulation (SMC) that accommodates missing data, non-normal
distributions, and facilitates model testing without extensive coding
(Y. Andre Wang and Rhemtulla 2021). For an overview of these solutions
and a discussion of analytical approaches, see Feng and Hancock (2023),
Jak et al. (2021), Nye (2022), and Yilin Andre Wang (2023).

However, it is a smart decision to run an SMC for the PA of your CFA
model using solutions that are consistent with the results' reproducible
and replicability. In this way, even analytical solutions that the
researcher may use as a starting point are recommended in the R
environment via the semTools packages (Jak et al. 2021) and semPower 2
(Jobst, Bader, and Moshagen 2023; Moshagen and Bader 2023). The first
option is compatible with the lavaan syntax and looks to be enough. The
second, albeit including SMC in some cases, has a more difficult syntax.

For detailed and tailored PA, especially in complex models or unique
study designs, the simsem package offers a robust solution, allowing for
the relaxation of traditional assumptions and supporting the use of
robust estimators. This package, which utilizes the familiar lavaan
syntax, simplifies the learning curve for researchers already accustomed
to SEM analyses, providing a user-friendly interface for conducting SMC
(Pornprasertmanit et al. 2022).

Publishing the sampling design and methodology enhances the
reproducibility and replicability of research, contributing to the
scientific community's collective understanding and validation of
measurement models (Jessica K. Flake, Pek, and Hehman 2017; Jessica Kay
Flake et al. 2022; Jessica Kay Flake and Fried 2020; Leite, Bandalos,
and Shen 2023). In the context of CFA, acknowledging the power
limitations of your study can signal potential concerns for the broader
inferences drawn from your research, emphasizing the importance of
external validity and the relevance of the outcomes over mere precision
(Leite, Bandalos, and Shen 2023).

\subsection{Pre-processing}\label{pre-processing}

Upon gathering and tabulating original data, ideally in non-binary
formats such as CSV, TXT, or JSON, the first step in data preprocessing
should be to eliminate responses from participants who have abandoned
the study. This identification often occurs at the end of preprocessing,
where these incomplete responses can offer insights into handling
missing data, outliers, and multicollinearity.

Incorporating control questions and measuring response time allows
researchers to further refine their dataset by excluding participants
who fail control items or complete the survey unusually quickly (Collier
2020). Calculating individual response variability (standard deviation)
can identify respondents who may not have engaged meaningfully with the
survey, indicated by minimal variation in their responses.

These preliminary data cleaning steps are fundamental yet frequently
overlooked in empirical research. They can significantly enhance data
quality before engaging in more complex statistical analyses. Visual and
descriptive examination of measurement model items is implicitly
beneficial for any statistical investigation and should be considered
standard practice.

While data transformation methods like linearization or normalization
are available, they are generally not necessary given the robust
estimation processes that can handle non-normal data (Brown 2015).
Parceling items is also discouraged due to its potential to obscure
underlying multidimensional structures (Brown 2015; Crede and Harms
2019).

Addressing missing data, outliers, and multicollinearity is critical.
Single imputation methods should be avoided as they underestimate error
variance and can lead to identification problems in your model (Enders
2023). For missing data under 5\%, the impact may be minimal, but for
higher rates, Full Information ML (FIML) or Multiple Imputation (MI)
should be utilized, with FIML often being the most straightforward and
effective choice for CFA (Brown 2015; Kline 2023).

FIML and MI are preferred for handling missing data due to their ability
to produce consistent and efficient parameter estimates under conditions
similar to MI (Enders 2023; Kline 2023). FIML it can be adapted for
non-normal data using robust estimators (Brown 2015).

Calculating the Variance Inflation Factor (VIF) helps identify items
with problematic multicollinearity (VIF \textgreater{} 10), which should
be addressed to prevent model convergence issues and misinterpretations
(Kline 2016; Whittaker and Schumacker 2022). Reflective constructs in
CFA require some level of item correlation but not to the extent that it
causes statistical or validity concerns.

Consider multivariate outliers rather than univariate ones, identifying
and assessing their exclusion based on sample characteristics. Reporting
all data cleaning processes, including any loss of items and strategies
for assessing respondent engagement, is crucial for transparency.
Additionally, documenting signs of multicollinearity and the software or
packages used (with versions) enhances the reproducibility and
credibility of the research (Jessica Kay Flake and Fried 2020; Jackson,
Gillaspy, and Purc-Stephenson 2009).

Finally, making raw data public adheres to the principles of open
science, promoting transparency and allowing for independent validation
of research findings (Crede and Harms 2019; Jessica Kay Flake et al.
2022; Jessica Kay Flake and Fried 2020). This practice not only
contributes to the scientific community's collective knowledge base but
also reinforces the integrity and reliability of the research conducted.

\subsection{Estimation Process}\label{estimation-process}

In CFA with ordinal items, such as those involving Likert-type scales
with up to five points, Rogers (2024) advocates for the use of
estimators from the Ordinary Least Squares (OLS) family. Specifically,
for smaller samples, the recommendation is to utilize the Unweighted
Least Squares (ULS) in its robust form (RULS), and for larger samples,
the Diagonally Weighted Least Squares (DWLS) in its robust version
(RDWLS), citing substantial supporting research.

Despite this, empirical evidence (Rhemtulla, Brosseau-Liard, and Savalei
2012; Robitzsch 2022) and theoretical considerations (Robitzsch 2020)
suggest that treating ordinal data as continuous can yield acceptable
outcomes when the response options number five or more. Particularly
with 6-7 categories, comparisons between methods under various
conditions reveal little difference, and it is recommended to use a
greater number of response alternatives (≥5) to enhance the power for
detecting model misspecifications (Maydeu-Olivares, Fairchild, and Hall
2017).

The ML estimator, noted for its robustness to minor deviations from
normality (Brown 2015), is further improved by using robust versions
like MLR (employing Huber-White standard errors and Yuan-Bentler scaled
\(\chi^2\). This adjustment allows for generating robust standard errors
and adjusted test statistics, with MLR offering extensive applicability
including in scenarios of missing data (RFIML) or where data breaches
the independence of observations assumption (Brown 2015; Rosseel 2012).
Comparative empirical studies have supported the effectiveness of MLR
against alternative estimators (Bandalos 2014; Holgado-Tello,
Morata-Ramirez, and García 2016; Li 2016; Nalbantoğlu-Yılmaz 2019; Yang
and Liang 2013; Yang-Wallentin, Jöreskog, and Luo 2010).

Researchers are advised to carefully describe and justify the chosen
estimation method based on the data characteristics and the specific
model being evaluated (Crede and Harms 2019). It is also critical to
report any estimation challenges encountered, such as algorithm
non-convergence or model misidentification (Nye 2022). In case of
estimation difficulties, alternative approaches like MLM estimators
(employing robust standard errors and Satorra-Bentler scaled \(\chi^2\))
or the default ML with non-parametric bootstrapping, as proposed by
Bollen-Stine, can be considered. This latter approach is also capable of
accommodating missing data (Brown 2015; Kline 2023).

Additionally, it is important to clarify whether the variance of a
marker variable was fixed (=1) to scale the latent variables (Jackson,
Gillaspy, and Purc-Stephenson 2009), and to provide both standardized
and unstandardized parameter estimates (Nye 2022). These steps are
crucial for ensuring transparency, reproducibility, and the ability to
critically assess the validity of the CFA results.

\subsection{Model Fit}\label{model-fit}

In conducting CFA with ordinal items, such as Likert-type scales, it's
crucial to approach model evaluation with nuance and avoid reliance on
rigid cutoff values for fit indices. Adhering strictly to traditional
cutoffs -- whether more conservative (e.g., SRMR ≤ .06, RMSEA ≤ .06, CFI
≥ .95) or less conservative (e.g., RMSEA ≤ .08, CFI ≥ .90, SRMR ≤ .08)
-- should not be the sole criterion for model acceptance (Xia and Yang
2019). The origins of these thresholds are in simulation studies with
specific configurations (up to three factors, fifteen items, factor
loadings between 0.7 and 0.8) (West et al. 2023), and may not
universally apply due to the variance in the number of items, factors,
model degrees of freedom, misfit types, and presence of missing data
(Groskurth, Bluemke, and Lechner 2023; Niemand and Mai 2018; West et al.
2023).

Evaluation of global fit indices (SRMR, RMSEA, CFI) should be done in a
collective manner, rather than fixating on any single index. A deviation
from traditional cutoffs warrants further investigation into whether the
discrepancy is attributable to data characteristics or limitations of
the index, rather than indicating a fundamental model misspecification
(Nye 2022). Interpreting fit indices as effect sizes can offer a more
meaningful assessment of model fit, aligning with their original
conceptualization (McNeish and Wolf 2023a; McNeish 2023b).

The SRMR is noted for its robustness across various conditions,
including non-normality and different measurement levels of items.
Pairing SRMR with CFI can help balance Type I and Type II errors, but
reliance on alternative indices may increase the risk of Type I error
(Mai, Niemand, and Kraus 2021; Niemand and Mai 2018).

Emerging methods like the Dynamic Fit Index (DFI) and Flexible Cutoffs
(FCO) offer tailored approaches to evaluating global fit. DFI, based on
simulation, provides model-specific cutoff points, adjusting simulations
to match the empirical model's characteristics (McNeish 2023a; McNeish
and Wolf 2023b; McNeish and Wolf 2023a). FCO, while not requiring
identification of a misspecified model like DFI, conservatively defines
misfit, shifting focus from approximate to accurate fit (McNeish and
Wolf 2023b).

For those hesitant to delve into simulation-based methods, Equivalence
Testing (EQT) presents an alternative. EQT aligns with the analytical
mindset of PA and incorporates DFI principles, challenging the
conventional hypothesis testing framework by considering model
specification and misspecification size control (Yuan et al. 2016).

When addressing reliability, Cronbach's Alpha should not be the default
measure due to its limitations. Instead, consider McDonald's Omega or
the Greatest Lower Bound (GLB) for a more accurate reliability
assessment within the CFA context (Bell, Chalmers, and Flora 2023; Cho
2022; Dunn, Baguley, and Brunsden 2014; Flora 2020; Goodboy and Martin
2020; Green and Yang 2015; Hayes and Coutts 2020; Kalkbrenner 2023;
McNeish 2018; Trizano-Hermosilla and Alvarado 2016).

Before modifying the model, first check for Heywood instances, which are
standardized factor loadings greater than one or negative variances (Nye
2022) and document the chosen cutoffs for evaluation. Tools and
resources like ShinyApp for DFI and the FCO package in R can facilitate
the application of these advanced methodologies (McNeish and Wolf 2023a;
Mai, Niemand, and Kraus 2021; Niemand and Mai 2018). Always report
corrected chi-square and degrees of freedom, alongside a minimum of
three global fit indices (RMSEA, CFI, SRMR) and local fit measures to
provide a comprehensive view of model fit and adjustment decisions
(Crede and Harms 2019; Jessica Kay Flake and Fried 2020).

\subsection{Model Comparisons and
Modifications}\label{model-comparisons-and-modifications}

Researchers embarking on CFA should avoid prematurely committing to a
specific factor structure without thoroughly evaluating and comparing
alternate configurations. It's advisable to consider various potential
structures early in the study design, ensuring the selected model is
based on its merits relative to competing theories (Jackson, Gillaspy,
and Purc-Stephenson 2009). Since models are inherently approximations of
reality, adopting the most effective ``working hypothesis'' is a dynamic
process, contingent on ongoing assessments against emerging alternatives
(Preacher and Yaremych 2023).

Good models are characterized not only by their interpretability,
simplicity, and generalizability but notably by their capacity to
surpass competing models in critical aspects. This competitive advantage
frames the selected theory as the prevailing hypothesis until a more
compelling alternative is identified (Preacher and Yaremych 2023).

The evaluation of model fit should extend beyond isolated assessments
using fit indices. A comprehensive approach involves comparing multiple
models, each grounded in substantiated theories, to discern the most
accurate representation of the underlying structure. This comparative
analysis is preferred over singular model evaluations, fostering a more
holistic understanding of the phenomena under study (Preacher and
Yaremych 2023).

Uniform application of models across the same dataset, utilizing
identical software and sample size, ensures consistency in the
researcher's analytical freedom, mitigating the risk of results
manipulation. This standardized approach underpins a more rigorous and
transparent investigative process (Preacher and Yaremych 2023).

Model selection is instrumental in pinpointing the most effective
explanatory framework for the observed phenomena, enabling the dismissal
of less performance models while retaining promising ones for further
exploration. This methodological flexibility enhances the depth of
analysis, contributing to the advancement of knowledge within the social
sciences (Preacher and Yaremych 2023).

Adjustments to a model, particularly in response to unsatisfactory fit
indices, should be theoretically grounded and reflective of findings
from prior research. Blind adherence to a pre-established model or
making hasty modifications can adversely affect the structural model's
integrity. Thoughtful adjustments, potentially revisiting exploratory
factor analysis (EFA) or considering Exploratory SEM (ESEM) for
cross-loadings representation, are preferable to drastic changes that
might shift the study from confirmatory to exploratory research (Brown
2023; Jessica K. Flake, Pek, and Hehman 2017; Jackson, Gillaspy, and
Purc-Stephenson 2009; Crede and Harms 2019).

All modifications to the measurement model, especially those enhancing
model fit, must be meticulously documented to maintain transparency and
support reproducibility (Jessica Kay Flake and Fried 2020). Openly
reporting these adjustments, including item exclusions and inter-item
correlations, is vital for the scientific integrity of the research (Nye
2022; Jessica Kay Flake et al. 2022).

Regarding model comparison and selection, traditional fit indices (SRMR,
RMSEA, CFI) have limitations for direct model comparisons. Adjusted
chi-square tests and information criteria like AIC and BIC are more
suitable for this purpose, balancing model fit and parsimony. These
criteria, however, should be applied with an understanding of their
constraints and complemented by theoretical judgment to inform model
selection decisions (Preacher and Yaremych 2023; Brown 2015; Huang 2017;
Lai 2020, 2021).

Ultimately, model selection in SEM is a nuanced process, blending
empirical evidence with theoretical insights. Researchers are encouraged
to leverage a range of models based on theoretical foundations, ensuring
that the eventual model selection is not solely determined by
statistical criteria but is also informed by substantive theory and
expertise (Preacher and Yaremych 2023). This balanced approach
underscores the importance of theory-driven research in the social
sciences, guiding the interpretation and application of findings derived
from chosen models.

\bookmarksetup{startatroot}

\section{Executable Manuscript}\label{executable-manuscript}

\subsection{Measurement Model
Selection}\label{measurement-model-selection-1}

\subsection{Power Analysis}\label{power-analysis-1}

\subsection{Pre-processing}\label{pre-processing-1}

\subsection{Estimation Process}\label{estimation-process-1}

\subsection{Model Fit}\label{model-fit-1}

\subsection{Model Comparisons and
Modifications}\label{model-comparisons-and-modifications-1}

\bookmarksetup{startatroot}

\section{Final Considerations}\label{final-considerations}

\bookmarksetup{startatroot}

\section*{References}\label{references}
\addcontentsline{toc}{section}{References}

\markboth{References}{References}

\phantomsection\label{refs}
\begin{CSLReferences}{1}{0}
\bibitem[\citeproctext]{ref-aera1999}
AERA, APA, and NCME. 1999. {``Standards for {Educational} and
{Psychological Testing}.''} Washington: American Educational Research
Association, American Psychological Association, \& National Council on
Measurement in Education.

\bibitem[\citeproctext]{ref-aera2014}
---------. 2014. {``Standards for {Educational} and {Pshychological
Testing}.''} Washington: American Educational Research Association,
American Psychological Association \& National Council on Measurement in
Education.

\bibitem[\citeproctext]{ref-arbuckle2019}
Arbuckle, J. L. 2019. {``Amos.''} Chicago: IBM Corp.

\bibitem[\citeproctext]{ref-bandalos2014}
Bandalos, Deborah L. 2014. {``Relative {Performance} of {Categorical
Diagonally Weighted Least Squares} and {Robust Maximum Likelihood
Estimation}.''} \emph{Structural Equation Modeling} 21 (1): 102--16.
\url{https://doi.org/10.1080/10705511.2014.859510}.

\bibitem[\citeproctext]{ref-bandalos2018}
---------. 2018. \emph{Measurement Theory and Applications for the
Social Sciences}. New York: Guilford Press.

\bibitem[\citeproctext]{ref-bell2023}
Bell, Stephanie M., R. Philip Chalmers, and David B. Flora. 2023. {``The
{Impact} of {Measurement Model Misspecification} on {Coefficient Omega
Estimates} of {Composite Reliability}.''} \emph{Educational and
Psychological Measurement}, 1--36.
\url{https://doi.org/10.1177/00131644231155804}.

\bibitem[\citeproctext]{ref-bentler2020}
Bentler, Peter M., and Erik Wu. 2020. {``{EQS} 6.4 for {Windows}.''}
Multivariate Software, Inc. \url{https://mvsoft.com}.

\bibitem[\citeproctext]{ref-brown2015}
Brown, Timothy A. 2015. \emph{Confirmatory {Factor Analysis} for
{Applied Research}}. New York: The Guilford Press.

\bibitem[\citeproctext]{ref-brown2023}
---------. 2023. {``Confirmatory {Factor Analysis}.''} In \emph{Handbook
of {Structural Equation Modeling}}, edited by Rick H. Hoyle. New York:
The Guilford Press.

\bibitem[\citeproctext]{ref-cho2022}
Cho, Eunseong. 2022. {``Reliability and {Omega Hierarchical} in
{Multidimensional Data}: {A Comparison} of {Various Estimators}.''}
\emph{Psychological Methods}. \url{https://doi.org/10.1037/met0000525}.

\bibitem[\citeproctext]{ref-cohen2022}
Cohen, Ronald Jay, Joel W. Schneider, and Renée M. Tobin. 2022.
\emph{Psychological {Testing} and {Assessment}: {An Introduction} to
{Test} and {Measurement}}. New York: McGraw Hill LLC.

\bibitem[\citeproctext]{ref-collier2020}
Collier, Joel E. 2020. \emph{Applied {Structural Equation Modeling Using
AMOS}: {Basic} to {Advanced Techniques}}. New York: Routledge.

\bibitem[\citeproctext]{ref-crede2019}
Crede, Marcus, and Peter Harms. 2019. {``Questionable Research Practices
When Using Confirmatory Factor Analysis.''} \emph{Journal of Managerial
Psychology} 34 (1): 18--30.
\url{https://doi.org/10.1108/JMP-06-2018-0272}.

\bibitem[\citeproctext]{ref-davvetas2020}
Davvetas, Vasileios, Adamantios Diamantopoulos, Ghasem Zaefarian, and
Christina Sichtmann. 2020. {``Ten Basic Questions about Structural
Equations Modeling You Should Know the Answers to -- {But} Perhaps You
Don't.''} \emph{Industrial Marketing Management} 90 (October): 252--63.
\url{https://doi.org/10.1016/j.indmarman.2020.07.016}.

\bibitem[\citeproctext]{ref-dunn2014}
Dunn, Thomas J., Thom Baguley, and Vivienne Brunsden. 2014. {``From
Alpha to Omega: {A} Practical Solution to the Pervasive Problem of
Internal Consistency Estimation.''} \emph{British Journal of Psychology}
105 (3): 399--412. \url{https://doi.org/10.1111/bjop.12046}.

\bibitem[\citeproctext]{ref-enders2023}
Enders, Craig. 2023. {``Fitting Structural {Equation Models} with
{Missing} Data.''} In \emph{Handbook of {Structural Equation Modeling}},
edited by Rick H. Hoyle. New York: The Guilford Press.

\bibitem[\citeproctext]{ref-feng2023}
Feng, Yi, and Gregory R. Hancock. 2023. {``Power {Analysis} Within a
{Structural Equation Modeling Framework}.''} In \emph{Handbook of
{Structural Equation Modeling}}, edited by Rick H. Hoyle. New York: The
Guilford Press.

\bibitem[\citeproctext]{ref-flake2022}
Flake, Jessica Kay, Ian J. Davidson, Octavia Wong, and Jolynn Pek. 2022.
{``Construct Validity and the Validity of Replication Studies: {A}
Systematic Review.''} \emph{American Psychologist} 77 (4): 576--88.
\url{https://doi.org/10.1037/amp0001006}.

\bibitem[\citeproctext]{ref-flake2020}
Flake, Jessica Kay, and Eiko I. Fried. 2020. {``Measurement
{Schmeasurement}: {Questionable Measurement Practices} and {How} to
{Avoid Them}.''} \emph{Advances in Methods and Practices in
Psychological Science} 3 (4): 456--65.
\url{https://doi.org/10.1177/2515245920952393}.

\bibitem[\citeproctext]{ref-flake2017}
Flake, Jessica K., Jolynn Pek, and Eric Hehman. 2017. {``Construct
{Validation} in {Social} and {Personality Research}: {Current Practices}
and {Recommendations}.''} \emph{Social Psychological and Personality
Science} 8 (4): 370--78. \url{https://doi.org/10.1177/1948550617693063}.

\bibitem[\citeproctext]{ref-flora2020}
Flora, David B. 2020. {``Your {Coefficient Alpha Is Probably Wrong}, but
{Which Coefficient Omega Is Right}? {A Tutorial} on {Using R} to {Obtain
Better Reliability Estimates}.''} \emph{Advances in Methods and
Practices in Psychological Science} 3 (4): 484--501.
\url{https://doi.org/10.1177/2515245920951747}.

\bibitem[\citeproctext]{ref-flora2017}
Flora, David B., and Jessica K. Flake. 2017. {``The Purpose and Practice
of Exploratory and Confirmatory Factor Analysis in Psychological
Research: {Decisions} for Scale Development and Validation.''}
\emph{Canadian Journal of Behavioural Science} 49 (2): 78--88.
\url{https://doi.org/10.1037/cbs0000069}.

\bibitem[\citeproctext]{ref-fox2022}
Fox, John. 2022. {``Sem: {Structural Equation Modeling}.''} R package.
\url{https://cran.r-project.org/web/packages/sem/}.

\bibitem[\citeproctext]{ref-furr2021}
Furr, Michael R. 2021. \emph{Psychometrics: {An Introduction}}. SAGE
Publications.

\bibitem[\citeproctext]{ref-gilroy2019}
Gilroy, Shawn P., and Brent A. Kaplan. 2019. {``Furthering {Open
Science} in {Behavior Analysis}: {An Introduction} and {Tutorial} for
{Using GitHub} in {Research}.''} \emph{Perspectives on Behavior Science}
42 (3): 565--81. \url{https://doi.org/10.1007/s40614-019-00202-5}.

\bibitem[\citeproctext]{ref-goodboy2020}
Goodboy, Alan K., and Matthew M. Martin. 2020. {``Omega over Alpha for
Reliability Estimation of Unidimensional Communication Measures.''}
\emph{Annals of the International Communication Association} 44 (4):
422--39. \url{https://doi.org/10.1080/23808985.2020.1846135}.

\bibitem[\citeproctext]{ref-green1997}
Green, Samuel B., Theresa M. Akey, Kandace K. Fleming, Scott L.
Hershberger, and Janet G. Marquis. 1997. {``Effect of the Number of
Scale Points on Chi-Square Fit Indices in Confirmatory Factor
Analysis.''} \emph{Structural Equation Modeling: A Multidisciplinary
Journal} 4 (2): 108--20.
\url{https://doi.org/10.1080/10705519709540064}.

\bibitem[\citeproctext]{ref-green2015}
Green, Samuel B., and Yanyun Yang. 2015. {``Evaluation of
{Dimensionality} in the {Assessment} of {Internal Consistency
Reliability}: {Coefficient Alpha} and {Omega Coefficients}.''}
\emph{Educational Measurement: Issues and Practice} 34 (4): 14--20.
\url{https://doi.org/10.1111/emip.12100}.

\bibitem[\citeproctext]{ref-groskurth2023}
Groskurth, Katharina, Matthias Bluemke, and Clemens M. Lechner. 2023.
{``Why We Need to Abandon Fixed Cutoffs for Goodness-of-Fit Indices:
{An} Extensive Simulation and Possible Solutions.''} \emph{Behavior
Research Methods}, August.
\url{https://doi.org/10.3758/s13428-023-02193-3}.

\bibitem[\citeproctext]{ref-hair2022}
Hair, Joseph F., Tomas M. G. Hult, Christian M. Ringle, and Marko
Sarstedt. 2022. \emph{A {Primer} on {Partial Least Squares Structural
Equation Modeling} ({PLS-SEM})}. Thousand Oaks: Sage Publications.

\bibitem[\citeproctext]{ref-hair2017}
Hair, Joseph F., Marko Sarstedt, Christian Ringle, and Siegfried P.
Gudergan. 2017. \emph{Advanced {Issues} in {Partial Least Squares
Structural Equation Modeling}}. London: SAGE Publications, Inc.

\bibitem[\citeproctext]{ref-harrington2009}
Harrington, Donna. 2009. \emph{Confirmatory {Factor Analysis}}. New
York: Oxford University Press.

\bibitem[\citeproctext]{ref-hayes2020}
Hayes, Andrew F., and Jacob J. Coutts. 2020. {``Use {Omega Rather} Than
{Cronbach}'s {Alpha} for {Estimating Reliability}. {But}{\ldots{}}.''}
\emph{Communication Methods and Measures} 14 (1): 1--24.
\url{https://doi.org/10.1080/19312458.2020.1718629}.

\bibitem[\citeproctext]{ref-henseler2021}
Henseler, Jörg. 2021. \emph{Composite-{Based Structural Equation
Modeling}: {Analyzing Latent} and {Emergent Variables}}. New York: The
Guilford Press.

\bibitem[\citeproctext]{ref-holgado-tello2016}
Holgado-Tello, F., M. Morata-Ramirez, and M. García. 2016. {``Robust
{Estimation Methods} in {Confirmatory Factor} {Analysis} of {Likert
Scales}: {A Simulation Study}.''} \emph{International Review of Social
Sciences and Humanities} 11 (2): 80--96.

\bibitem[\citeproctext]{ref-hoyle2023cap1}
Hoyle, Rick H. 2023. {``Structural {Equation Modeling}: {An
Overview}.''} In \emph{Handbook of {Structural Equation Modeling}},
edited by Rick H. Hoyle. New Yoirk: Guilford Press.

\bibitem[\citeproctext]{ref-huang2017}
Huang, Po-Hsien. 2017. {``Asymptotics of {AIC}, {BIC}, and {RMSEA} for
{Model Selection} in {Structural Equation Modeling}.''}
\emph{Psychometrika} 82 (2): 407--26.
\url{https://doi.org/10.1007/s11336-017-9572-y}.

\bibitem[\citeproctext]{ref-hughes2018}
Hughes, David J. 2018. {``Psychometric {Validity}: {Establishing} the
{Accuracy} and {Appropriateness} of {Psychometric Measures}.''} In
\emph{The {Wiley Handbook} of {Psychometric Testing}: {A
Multidisciplinary Reference} on {Survey}, {Scale} and {Test
Development}}, edited by Paul Irwing, Tom Booth, and David J. Hughes.
John Wiley \& Sons Ltd.

\bibitem[\citeproctext]{ref-jackson2009}
Jackson, Dennis L., J. Arthur Gillaspy, and Rebecca Purc-Stephenson.
2009. {``Reporting Practices in Confirmatory Factor Analysis: {An}
Overview and Some Recommendations.''} \emph{Psychological Methods} 14
(1). \url{https://doi.org/10.1037/a0014694}.

\bibitem[\citeproctext]{ref-jak2021}
Jak, Suzanne, Terrence D Jorgensen, Mathilde G E Verdam, Frans J Oort,
and Louise Elffers. 2021. {``Analytical Power Calculations for
Structural Equation Modeling: {A} Tutorial and {Shiny} App.''}
\emph{Behavior Research Mehods} 53: 1385--1406.
\url{https://doi.org/10.3758/s13428-020-01479-0/Published}.

\bibitem[\citeproctext]{ref-jasp2023}
JASP Team. 2023. {``{JASP}.''} {[}Computer Software{]}.
\url{https://jasp-stats.org/}.

\bibitem[\citeproctext]{ref-jobst2023}
Jobst, Lisa J., Martina Bader, and Morten Moshagen. 2023. {``A Tutorial
on Assessing Statistical Power and Determining Sample Size for
Structural Equation Models.''} \emph{Psychological Methods} 28 (1):
207--21. \url{https://doi.org/10.1037/met0000423}.

\bibitem[\citeproctext]{ref-joreskog2022}
Jöreskog, K. G., and D. Sörbom. 2022. {``{LISREL} 12 for {Windows}.''}
Scientific Software International, Inc.
\url{https://ssicentral.com/index.php/products/lisrel/}.

\bibitem[\citeproctext]{ref-kalkbrenner2023}
Kalkbrenner, Michael T. 2023. {``Alpha, {Omega}, and {H Internal
Consistency Reliability Estimates}: {Reviewing These Options} and {When}
to {Use Them}.''} \emph{Counseling Outcome Research and Evaluation} 14
(1): 77--88. \url{https://doi.org/10.1080/21501378.2021.1940118}.

\bibitem[\citeproctext]{ref-kathawalla2021}
Kathawalla, Ummul-Kiram, Priya Silverstein, and Moin Syed. 2021.
{``Easing {Into Open Science}: {A Guide} for {Graduate Students} and
{Their Advisors}.''} \emph{Collabra: Psychology} 7 (1): 18684.
\url{https://doi.org/10.1525/collabra.18684}.

\bibitem[\citeproctext]{ref-klein2018}
Klein, Olivier, Tom E. Hardwicke, Frederik Aust, Johannes Breuer, Henrik
Danielsson, Alicia Hofelich Mohr, Hans IJzerman, Gustav Nilsonne, Wolf
Vanpaemel, and Michael C. Frank. 2018. {``A {Practical Guide} for
{Transparency} in {Psychological Science}.''} Edited by Michéle Nuijten
and Simine Vazire. \emph{Collabra: Psychology} 4 (1): 20.
\url{https://doi.org/10.1525/collabra.158}.

\bibitem[\citeproctext]{ref-kline2016}
Kline, Rex B. 2016. \emph{Principles and {Pratice} of {Structural
Equation Modeling}}. New York: The Guilford Press.

\bibitem[\citeproctext]{ref-kline2023}
---------. 2023. \emph{Principles and {Pratice} of {Structural Equation
Modeling}}. Fifth Edition. New York: The Guilford Press.

\bibitem[\citeproctext]{ref-kyriazos2018}
Kyriazos, Theodoros A. 2018. {``Applied {Psychometrics}: {Sample Size}
and {Sample Power Considerations} in {Factor Analysis} ({EFA}, {CFA})
and {SEM} in {General}.''} \emph{Psychology} 09 (08): 2207--30.
\url{https://doi.org/10.4236/psych.2018.98126}.

\bibitem[\citeproctext]{ref-lai2020nonnested}
Lai, Keke. 2020. {``Confidence {Interval} for {RMSEA} or {CFI Difference
Between Nonnested Models}.''} \emph{Structural Equation Modeling: A
Multidisciplinary Journal} 27 (1): 16--32.
\url{https://doi.org/10.1080/10705511.2019.1631704}.

\bibitem[\citeproctext]{ref-lai2021fit}
---------. 2021. {``Fit {Difference Between Nonnested Models Given
Categorical Data}: {Measures} and {Estimation}.''} \emph{Structural
Equation Modeling: A Multidisciplinary Journal} 28 (1): 99--120.
\url{https://doi.org/10.1080/10705511.2020.1763802}.

\bibitem[\citeproctext]{ref-leite2023}
Leite, Walter L., Deborah L. Bandalos, and Zuchao Shen. 2023.
{``Simulation {Methods} in {Structural Equation Modeling}.''} In
\emph{Handbook of {Structural Equation Modeling}}, edited by Rick H.
Hoyle. New York: The Guilford Press.

\bibitem[\citeproctext]{ref-li2016cfa}
Li, Cheng Hsien. 2016. {``Confirmatory Factor Analysis with Ordinal
Data: {Comparing} Robust Maximum Likelihood and Diagonally Weighted
Least Squares.''} \emph{Behavior Research Methods} 48 (3): 936--49.
\url{https://doi.org/10.3758/s13428-015-0619-7}.

\bibitem[\citeproctext]{ref-mai2021}
Mai, Robert, Thomas Niemand, and Sascha Kraus. 2021. {``A Tailored-Fit
Model Evaluation Strategy for Better Decisions about Structural Equation
Models.''} \emph{Technological Forecasting and Social Change} 173
(December): 121142.
\url{https://doi.org/10.1016/j.techfore.2021.121142}.

\bibitem[\citeproctext]{ref-martins2021}
Martins, Henrique Castro. 2021. {``Tutorial-{Articles}: {The Importance}
of {Data} and {Code Sharing}.''} \emph{Revista de Administra{ç}{ã}o
Contempor{â}nea} 25 (1): e200212.
\url{https://doi.org/10.1590/1982-7849rac2021200212}.

\bibitem[\citeproctext]{ref-maydeu-olivares2017a}
Maydeu-Olivares, Alberto, Amanda J. Fairchild, and Alexander G. Hall.
2017. {``Goodness of {Fit} in {Item Factor Analysis}: {Effect} of the
{Number} of {Response Alternatives}.''} \emph{Structural Equation
Modeling: A Multidisciplinary Journal} 24 (4): 495--505.
\url{https://doi.org/10.1080/10705511.2017.1289816}.

\bibitem[\citeproctext]{ref-mcneish2018}
McNeish, Daniel. 2018. {``Thanks Coefficient Alpha, {We}'ll Take It from
Here.''} \emph{Psychological Methods} 23 (3): 412--33.
\url{https://doi.org/10.1037/met0000144}.

\bibitem[\citeproctext]{ref-mcneish2023likert}
---------. 2023a. {``Dynamic {Fit Index Cutoffs} for {Factor Analysis}
with {Likert}, {Ordinal}, or {Binary Responses}.''} \emph{PsyArXiv
Preprints}. \url{https://doi.org/10.31234/osf.io/tp35s}.

\bibitem[\citeproctext]{ref-mcneish2023geral}
---------. 2023b. {``Generalizability of {Dynamic Fit Index},
{Equivalence Testing}, and {Hu} \& {Bentler Cutoffs} for {Evaluating
Fit} in {Factor Analysis}.''} \emph{Multivariate Behavioral Research} 58
(1): 195--219. \url{https://doi.org/10.1080/00273171.2022.2163477}.

\bibitem[\citeproctext]{ref-mcneishwolf2023cfa}
McNeish, Daniel, and Melissa G. Wolf. 2023a. {``Dynamic Fit Index
Cutoffs for Confirmatory Factor Analysis Models.''} \emph{Psychological
Methods} 28 (1): 61--88. \url{https://doi.org/10.1037/met0000425}.

\bibitem[\citeproctext]{ref-mcneishwolf2023dddf}
McNeish, Daniel, and Melissa Gordon Wolf. 2023b. {``Direct {Discrepancy
Dynamic Fit Index Cutoffs} for {Arbitrary Covariance Structure
Models}.''} Preprint. PsyArXiv.
\url{https://doi.org/10.31234/osf.io/4r9fq}.

\bibitem[\citeproctext]{ref-mendes-da-silva2023}
Mendes-Da-Silva, Wesley. 2023. {``What {Lectures} and {Research} in
{Business Management Need} to {Know About Open Science}.''}
\emph{Revista de Administra{ç}{ã}o de Empresas} 63 (4): e0000--0033.
\url{https://doi.org/10.1590/s0034-759020230408x}.

\bibitem[\citeproctext]{ref-moshagen2023}
Moshagen, Morten, and Martina Bader. 2023. {``{semPower}: {General}
Power Analysis for Structural Equation Models.''} \emph{Behavior
Research Methods}, November.
\url{https://doi.org/10.3758/s13428-023-02254-7}.

\bibitem[\citeproctext]{ref-muthen2023}
Muthén, L. K., and B. O. Muthén. 2023. {``Mplus Version 8.9 User's
Guide.''}

\bibitem[\citeproctext]{ref-nalbantoglu-yilmaz2019}
Nalbantoğlu-Yılmaz, Funda. 2019. {``Comparison of {Different Estimation
Methods Used} in {Confirmatory Factor Analyses} in {Non-Normal Data}: {A
Monte Carlo Study}.''} \emph{International Online Journal of Educational
Sciences} 11 (4). \url{https://doi.org/10.15345/iojes.2019.04.010}.

\bibitem[\citeproctext]{ref-neale2016}
Neale, Michael C., Michael D. Hunter, Joshua N. Pritikin, Mahsa Zahery,
Timothy R. Brick, Robert M. Kirkpatrick, Ryne Estabrook, Timothy C.
Bates, Hermine H. Maes, and Steven M. Boker. 2016. {``{OpenMx} 2.0:
{Extended Structural Equation} and {Statistical Modeling}.''}
\emph{Psychometrika} 81 (2): 535--49.
\url{https://doi.org/10.1007/s11336-014-9435-8}.

\bibitem[\citeproctext]{ref-niemand2018}
Niemand, Thomas, and Robert Mai. 2018. {``Flexible Cutoff Values for Fit
Indices in the Evaluation of Structural Equation Models.''}
\emph{Journal of the Academy of Marketing Science} 46 (6): 1148--72.
\url{https://doi.org/10.1007/s11747-018-0602-9}.

\bibitem[\citeproctext]{ref-nye2022}
Nye, Christopher D. 2022. {``Reviewer {Resources}: {Confirmatory Factor
Analysis}.''} \emph{Organizational Research Methods}, August,
109442812211205. \url{https://doi.org/10.1177/10944281221120541}.

\bibitem[\citeproctext]{ref-pilcher2023}
Pilcher, Nick, and Martin Cortazzi. 2023. {``'{Qualitative}' and
'Quantitative' Methods and Approaches Across Subject Fields:
Implications for Research Values, Assumptions, and Practices.''}
\emph{Quality \& Quantity}, September.
\url{https://doi.org/10.1007/s11135-023-01734-4}.

\bibitem[\citeproctext]{ref-pornprasertmanit2022}
Pornprasertmanit, Sunthud, Patrick Miller, Terrence D. Jorgensen, and
Quick Corbin. 2022. {``Simsem: {SIMulated Structural Equation
Modeling}.''} R package. \href{https://www.simsem.org}{www.simsem.org}.

\bibitem[\citeproctext]{ref-preacher2023}
Preacher, Kristopher J., and Haley E. Yaremych. 2023. {``Model
{Selection} in {Structural Equation Modeling}.''} In \emph{Handbook of
{Structural Equation Modeling}}, edited by Rick H. Hoyle. New York: The
Guilford Press.

\bibitem[\citeproctext]{ref-price2017}
Price, Larry R. 2017. \emph{Psychometric {Methods}: {Theory} into
{Practice}}. 1st Edition. Methodology in the Social Sciences. New York:
The Guilford Press.

\bibitem[\citeproctext]{ref-reeves2016}
Reeves, Todd D., and Gili Marbach-Ad. 2016. {``Contemporary Test
Validity in Theory and Practice: {A} Primer for Discipline-Based
Education Researchers.''} \emph{CBE Life Sciences Education} 15 (1).
\url{https://doi.org/10.1187/cbe.15-08-0183}.

\bibitem[\citeproctext]{ref-rhemtulla2012}
Rhemtulla, Mijke, Patricia É Brosseau-Liard, and Victoria Savalei. 2012.
{``When Can Categorical Variables Be Treated as Continuous? {A}
Comparison of Robust Continuous and Categorical {SEM} Estimation Methods
Under Suboptimal Conditions.''} \emph{Psychological Methods} 17 (3):
354--73. \url{https://doi.org/10.1037/a0029315}.

\bibitem[\citeproctext]{ref-ringle2022}
Ringle, Christian M., Sven Wende, and Jan Michael Becker. 2022.
{``{SmartPLS} 4.''} Oststeinbek: SmartPLS.
\url{https://www.smartpls.com}.

\bibitem[\citeproctext]{ref-rios2014}
Rios, Joseph, and Craig Wells. 2014. {``Validity Evidence Based on
Internal Structure.''} \emph{Psicothema} 26 (1): 108--16.
\url{https://doi.org/10.7334/psicothema2013.260}.

\bibitem[\citeproctext]{ref-robitzsch2020}
Robitzsch, Alexander. 2020. {``Why {Ordinal Variables Can} ({Almost})
{Always Be Treated} as {Continuous Variables}: {Clarifying Assumptions}
of {Robust Continuous} and {Ordinal Factor Analysis Estimation
Methods}.''} \emph{Frontiers in Education} 5 (October).
\url{https://doi.org/10.3389/feduc.2020.589965}.

\bibitem[\citeproctext]{ref-robitzsch2022}
---------. 2022. {``On the {Bias} in {Confirmatory Factor Analysis When
Treating Discrete Variables} as {Ordinal Instead} of {Continuous}.''}
\emph{Axioms} 11 (4). \url{https://doi.org/10.3390/axioms11040162}.

\bibitem[\citeproctext]{ref-rogers2022}
Rogers, Pablo. 2022. {``Best {Practices} for {Your Exploratory Factor
Analysis}: {A Factor Tutorial}.''} \emph{Revista de Administra{ç}{ã}o
Contempor{â}nea} 26 (6).
\url{https://doi.org/10.1590/1982-7849rac2022210085.en}.

\bibitem[\citeproctext]{ref-rogers2024}
---------. 2024. {``Best Practices for Your Confirmatory Factor
Analysis: {A JASP} and Lavaan Tutorial.''} \emph{Behavior Research
Methods}, March. \url{https://doi.org/10.3758/s13428-024-02375-7}.

\bibitem[\citeproctext]{ref-rosseel2012}
Rosseel, Yves. 2012. {``Lavaan: {An R} Package for Structural Equation
Modeling.''} \emph{Journal of Statistical Software} 48 (2): 1--36.
\url{https://doi.org/10.18637/jss.v048.i02}.

\bibitem[\citeproctext]{ref-schumacker2021}
Schumacker, Randall E., Stefanie A. Wind, and Lauren F. Holmes. 2021.
{``Resources for {Identifying Measurement Instruments} for {Social
Science Research}.''} \emph{Measurement: Interdisciplinary Research and
Perspectives} 19 (4): 250--57.
\url{https://doi.org/10.1080/15366367.2021.1950486}.

\bibitem[\citeproctext]{ref-shek2014}
Shek, Daniel T. L., and Lu Yu. 2014. {``Use of Structural Equation
Modeling in Human Development Research.''} \emph{International Journal
on Disability and Human Development} 13 (2): 157--67.
\url{https://doi.org/10.1515/ijdhd-2014-0302}.

\bibitem[\citeproctext]{ref-sireci2013}
Sireci, Stephen G., and Tia Sukin. 2013. {``Test Validity.''} In
\emph{{APA} Handbook of Testing and Assessment in Psychology, {Vol}. 1:
{Test} Theory and Testing and Assessment in Industrial and
Organizational Psychology.}, 61--84. Washington: American Psychological
Association. \url{https://doi.org/10.1037/14047-004}.

\bibitem[\citeproctext]{ref-jamovi2023}
The jamovi project. 2023. {``Jamovi.''} {[}Computer Software{]}.
\url{https://www.jamovi.org}.

\bibitem[\citeproctext]{ref-trizano-hermosilla2016}
Trizano-Hermosilla, Italo, and Jesús M. Alvarado. 2016. {``Best
Alternatives to {Cronbach}'s Alpha Reliability in Realistic Conditions:
{Congeneric} and Asymmetrical Measurements.''} \emph{Frontiers in
Psychology} 7 (MAY). \url{https://doi.org/10.3389/fpsyg.2016.00769}.

\bibitem[\citeproctext]{ref-urbina2014}
Urbina, Susana. 2014. \emph{Essentials of {Psychological Testing}}.
Hoboken, New Jersey: John Wiley \& Sons.

\bibitem[\citeproctext]{ref-wang2021}
Wang, Y. Andre, and Mijke Rhemtulla. 2021. {``Power {Analysis} for
{Parameter Estimation} in {Structural Equation Modeling}: {A Discussion}
and {Tutorial}.''} \emph{Advances in Methods and Practices in
Psychological Science} 4 (1): 1--17.
\url{https://doi.org/10.1177/2515245920918253}.

\bibitem[\citeproctext]{ref-wang2023}
Wang, Yilin Andre. 2023. {``How to {Conduct Power Analysis} for
{Structural Equation Models}: {A Practical Primer}.''} Preprint.
PsyArXiv. \url{https://doi.org/10.31234/osf.io/4n3uk}.

\bibitem[\citeproctext]{ref-west2023}
West, Stephen G., Wei Wu, Daniel McNeish, and Andrea Savord. 2023.
{``Model {Fit} in {Structural Equation Modeling}.''} In \emph{Handbook
of {Structural Equation Modeling}}, edited by Rick H. Hoyle. New York:
The Guilford Press.

\bibitem[\citeproctext]{ref-westland2010}
Westland, Christopher J. 2010. {``Lower Bounds on Sample Size in
Structural Equation Modeling.''} \emph{Electronic Commerce Research and
Applications} 9 (6). \url{https://doi.org/10.1016/j.elerap.2010.07.003}.

\bibitem[\citeproctext]{ref-whittaker2022}
Whittaker, Tiffany A., and Randall E. Schumacker. 2022. \emph{A
{Beginner}'s {Guide} to {Structural Equation Modeling}}. Fifth Edition.
New York: Routledge Taylor \& Francis Group.

\bibitem[\citeproctext]{ref-widodo2018}
Widodo, Estu. 2018. {``Some {Notes} on the {Contemporary Views} of
{Validity} in {Psychological} and {Educational Assessment}.''}
\emph{Advances in Social Science, Education and Humanities Research}
231: 732--34. \url{https://doi.org/10.3968/8877}.

\bibitem[\citeproctext]{ref-wilkinson2016}
Wilkinson, Mark D., Michel Dumontier, IJsbrand Jan Aalbersberg,
Gabrielle Appleton, Myles Axton, Arie Baak, Niklas Blomberg, et al.
2016. {``The {FAIR Guiding Principles} for Scientific Data Management
and Stewardship.''} \emph{Scientific Data} 3 (1): 160018.
\url{https://doi.org/10.1038/sdata.2016.18}.

\bibitem[\citeproctext]{ref-xia2018}
Xia, Yan, and Yanyun Yang. 2018. {``The {Influence} of {Number} of
{Categories} and {Threshold Values} on {Fit Indices} in {Structural
Equation Modeling} with {Ordered Categorical Data}.''}
\emph{Multivariate Behavioral Research} 53 (5): 731--55.
\url{https://doi.org/10.1080/00273171.2018.1480346}.

\bibitem[\citeproctext]{ref-xia2019}
---------. 2019. {``{RMSEA}, {CFI}, and {TLI} in Structural Equation
Modeling with Ordered Categorical Data: {The} Story They Tell Depends on
the Estimation Methods.''} \emph{Behavior Research Methods} 51 (1):
409--28. \url{https://doi.org/10.3758/s13428-018-1055-2}.

\bibitem[\citeproctext]{ref-yang2013}
Yang, Yanyun, and Xinya Liang. 2013. {``Confirmatory Factor Analysis
Under Violations of Distributional and Structural Assumptions.''}
\emph{Int. J. Quantitative Research in Education} 1 (1): 61--84.

\bibitem[\citeproctext]{ref-yang-wallentin2010}
Yang-Wallentin, Fan, Karl G. Jöreskog, and Hao Luo. 2010.
{``Confirmatory Factor Analysis of Ordinal Variables with Misspecified
Models.''} \emph{Structural Equation Modeling} 17 (3): 392--423.
\url{https://doi.org/10.1080/10705511.2010.489003}.

\bibitem[\citeproctext]{ref-yuan2016}
Yuan, Ke Hai, Wai Chan, George A. Marcoulides, and Peter M. Bentler.
2016. {``Assessing {Structural Equation Models} by {Equivalence Testing
With Adjusted Fit Indexes}.''} \emph{Structural Equation Modeling} 23
(3): 319--30. \url{https://doi.org/10.1080/10705511.2015.1065414}.

\bibitem[\citeproctext]{ref-zhang2018}
Zhang, Zhiyong, and Ke-Hai Yuan. 2018. \emph{Practical Statistical Power
Analysis Using {Webpower} and {R}}. Indiana: ISDSA Press.
\url{https://doi.org/10.35566/power}.

\end{CSLReferences}



\end{document}
